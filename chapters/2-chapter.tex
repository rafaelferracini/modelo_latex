% ----------------------------------------------------------
\chapter{Análise gráfica}\label{cap:analise_grafica}
% ----------------------------------------------------------
\section{Linearização}
Tomando a equação horária da posição do movimento retilíneo uniforme
\begin{equation}\label{eq: funcao_horaria}
    x(t) = x_0 + v_x t
\end{equation}

Estamos interessados em tomar $t$ como variável dependente e $x$ como variável independente. Isolando t na \autoref{eq: funcao_horaria} temos
\begin{equation*}
    t = \frac{1}{v_x}x - \frac{x_0}{v_x}
\end{equation*}

Conseguimos então linearizar com uma equação da forma $Y = A + BX$ tomando
\begin{equation}\label{eq: linearizacao}
    t = \frac{1}{v_x}x - \frac{x_0}{v_x} \Rightarrow \left\{ \begin{array}{ll}
        Y = t &  \\
        X = x &  \\
        A = -\frac{x_0}{v_x} & \\
        B = \frac{1}{v_x}
    \end{array}\right.
\end{equation}

\section{Gráfico}
Com o auxílio do programa SciDAVis podemos plotar os dados da \autoref{table:experimento1} e obter a regressão linear com os respectivos coeficientes linear ($A = -0,629 \pm 0,0332712893656291$) e angular ($B =  0,0208848484848485 \pm 0,000288770195295941$) e seus erros, explicitado na \autoref{fig:grafico}.
\begin{figure}[htb]
	\caption{\label{fig:grafico}Gráfico do tempo em função do deslocamento}
	\begin{center}
		\includegraphics[width=10cm]{images/experimento5_grafico.png}
	\end{center}
	\fonte{Autoral.}
\end{figure}
